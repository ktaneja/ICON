\section{Temporal Specifications}
\label{sec:temporal}

In this section we describe various classes of temporal constraints using formal temporal logic.
In computer science, temporal Logic is a system of rules and symbols that are used to express and reason about prepositions pertaining to time.
We next present an extension to the temporal logic for API method invocation related rules proposed by Lo et al.~\cite{lo2009mining}

\textbf{Method Invocation}: 
We adapt the definition of events proposed by Lo et al.~\cite{lo2009mining} as method invocation.
They define event as \textit{``any concrete or abstract representation of a program state''}. 
We propose the notion of method invocation analogous to the events,
where each method invocation is represented as a tuple which is an ordered set of strings. 
A method invocation is captured as the Tuple <$T$><\CodeIn{str1, str2, str3}...>, where \CodeIn{str1, str2, str3}... are string values,
that belong to set of all strings.
The method invocation tuple <$T$> captures information about the return value, enclosing type, name, and the parameter types of the method.
Therefore a tuple <$T$><\CodeIn{String, Bar, foo, Integer, Integer}> represents a method \CodeIn{foo} enclosed by the type \CodeIn{Bar},
that accepts two parameters of type \CodeIn{Integer} and a return value of type \CodeIn{String}. 
For better readability mnemonic representation of the preceding tuple is \CodeIn{String<-Bar.foo(Integer, Integer);}



\textbf{Method Invocation Predicates}: We define method predicate as a predicate $E$ over method invocations.
We reuse the notations proposed by Lo et al.~\cite{lo2009mining} $m \vdash \xi$ to denote that method invocation $m$ satisfies method predicate $\xi$.
In context of \tool, method predicates combines a method name and the enclosing type with potential constraints on the return types and parameter types. 
The predicates are modeled after \textit{equality constraints predicates}, and are represented as \CodeIn{\$i=sym}, where \CodeIn{i} is a non-negative integer and \CodeIn{sym} is the primitive string value and defined as: a method invocation $m$ satisfies a predicates \CodeIn{\$i=sym} iff \CodeIn{m[i]=sym}. The simplest form of these predicates is of type \CodeIn{\$2 = sym} which refer to method invocations of name \CodeIn{sym}.

\textbf{Temporal Operators}: Temporal formulae (temporal constraints) are constructed by combining method invocation predicates using temporal operators. Lo et al.~\cite{lo2009mining}, proposed the following temporal operators    

\begin{enumerate}

\item \textit{Forward Eventual Operator ($\xi_1 \xrightarrow{*} \xi_2$)}:
occurrence of $\xi_1$ must be eventually followed by occurrence of $\xi_2$

\item \textit{Backward Eventual Operator: ($\xi_1 \xleftarrow{*} \xi_2$)}:
occurrence of $\xi_1$ must be preceded by occurrence of $\xi_2$

\item \textit{Forward Alternation Operator ($\xi_1 \xrightarrow{a} \xi_2$)}:
occurrence of $\xi_1$ must be eventually followed by occurrence of $\xi_2$ and $\xi_1$ cannot occur in between

\item \textit{Backward Alternation Operator ($\xi_1 \xleftarrow{a} \xi_2$)}:
occurrence of $\xi_1$ must be preceded by occurrence of $\xi_2$ and $\xi_1$ cannot occur in between
\end{enumerate}

We extend these operators by proposing two additional operators:
\begin{enumerate}
\item Forward Prohibition Operator ($\xi_1 \nrightarrow \xi_2$):
occurrence of $\xi_1$ cannot be followed by occurrence of $\xi_2$

\item Backward Prohibition Operator ($\xi_1 \nleftarrow \xi_2$):
occurrence of $\xi_1$ cannot be preceded by the occurrence of $\xi_2$
\end{enumerate}

The prohibition operators capture a different class of temporal constraints that cannot be expressed by the formulae constructed by previously proposed operators and their negations. For instance, the \textit{prohibition operators} on first glance seem to be negation of \textit{forward eventual operators} and \textit{backward eventual operators}. However, the negation of forward eventual operators is: not an invocation of e1 must be eventually followed by not an invocation of m2. In contrast, \textit{prohibition operators} either negates antecedent or consequent but not both.  

\textbf{Quantification}: Quantification is used to introduce constraints involving parameter and return types of a method invocation. For instance, consider the constraint any method in an enclosing type \CodeIn{Obj} must not be invoked after method \CodeIn{foo} in the \CodeIn{Obj}.

$\forall$ \CodeIn{m <m in Obj> !=<foo in Obj>}

In summary, we extended the formalism for temporal rules previously proposed by Lo et al.~\cite{lo2009mining} to express temporal constraints.
We obtain different classes of temporal constraints by varying: 1) the temporal operators allowed, b) the method-invocation predicates allowed, and 3) the degree of quantification allowed.