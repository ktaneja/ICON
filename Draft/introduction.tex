\section{Introduction}
\label{sec:introduction}

Application Programming Interfaces (APIs) facilitates software reuse
by providing standardized mechanism to access API components.
However, APIs have some constraints governing the proper use of the API that must be followed.
One such type of constraints are temporal constraints [2], which are the allowed sequences of invocations of methods from the API.
Non-compliance to such constraints will often result in faulty applications that are unreliable to use and may even result in financial losses.

Formal analysis tools, such as model checker and runtime verifiers 
can assist in detection of the violations of the temporal constraints
in API clients as defects~\cite{lee2012towards}.
These tools typically accept formal representation of the temporal constraints for detecting violations.
However, temporal constraints are typically described in natural language text of API documents.
Such documents are provided to client-code developers through an online access, or are shipped with the API code.
For a method under consideration, API document may describe both the constraints on the method parameters
as well as the temporal constraints in terms of other methods that must be invoked pre/post invoking that method.
%We observe that a non-trivial portion (roughly 12\%) of the sentences (constraint-describing sentences)
%in \amazonAPI\ documentation describe temporal constraints.

Although natural language API description can be manually converted into formal constraints,
manually identifying and writing formal constraints based on natural language text in API documents can be prohibitively time-consuming and error-prone~\cite{wu2013inferring,RubingerWEB10}. 
%For instance, Wu et al.~\cite{wu2013inferring} report that it took one person ten hours to browse the documentation of one method from a web API, even before the person attempted to formalize the constraints on the method. 
For instance, the PDF version of the documentation for \amazonAPI\footnote{{\small \url{http://awsdocs.s3.amazonaws.com/S3/latest/s3-api.pdf}}} spans 278 pages describing 51 methods.

\textit{The goal of this work is to assist developers construct API clients that comply with
	temporal constraints of the API through the inference and formalization of these constraints found in natural language API documents.}

First step towards formalizing temporal constraints is to identify the sentences describing such constraints.
A naive approach to identify the constraint sentences is to perform keyword-based search on API documents.
However, the effectiveness of such approach is limited by the quality of the keywords used.
Furthermore, sentences involving temporal constraints oftentimes may not have have uniquely identifiable keywords.

For instance, consider the sentences from \paypalAPI:
1) ``\textit{Use this call to complete a payment.}'' 
2) ``\textit{Use this method to refund a completed payment.}''
Sentence 1 indicates the temporal constraint that a payment must be completed before the refund call is initiated.
In contrast, sentence 2 is a descriptive statement about complete payment method.
Since these sentences are not significantly different in terms of words, a simple keyword-based search will fail to distinguish between the two. 
Another problem with keyword-based search is that the method names (and synonyms) are part of the keywords themselves, thus resulting in a large number of keywords to be searched.
The large number of keywords further negatively affects accuracy.

Even after a sentence has been identified as a constraint sentence, an approach has to infer the references to the method in the sentences, which may often not be implicit. Consider Sentence 2 in preceding paragraph. The phrase \textit{``completed payment''} refers to the ``Execute Payment'' method in the API.


To address these issues, we propose \tool: an approach based on Machine Learning (ML) and Natural Language Processing (NLP)
for identifying and inferring formal temporal constraints.
We propose to first employ ML for identifying temporal constraint sentences and then use NLP techniques to infer formal temporal constraints from the identified sentences. 

A ML based approach for identifying the temporal constraint sentences addresses the limitations of keyword-based search by automatically learning patterns of temporal constraint sentences.
Furthermore, we propose to use a combination of words, syntax (parts of speech) of a sentence, and semantics (relationship between words) of a sentence as the features for learning patterns.
This combination allows our approach to make a finer grained distinction between the example sentences described earlier.
To identify the phrases as method-invocation instances, we propose to build domain dictionaries systematically from API documents and generic English dictionaries.

In summary, \tool\ approach leverages natural language description of API's to infer temporal constraints of method invocations.
As our approach analyzes API documents in natural language, it can be reused independent of the programming language of an API library.
Additionally, our approach complements existing mining-based approaches~\cite{buse2012synthesizing, thummalapenta07parseweb, Wang:2013:MSR, Zhong:2009:MMR} that partially address the problem by mining for common usage patterns among client code reusing the API.
Our results indicate that \tool\ is effective in identifying temporal
constraint sentences (from over
4000 human-annotated API sentences) with the average precision, recall, and F-score
of 79.0\%, 60.0\%, and 65.0\%, respectively.
Furthermore, our evaluation also demonstrate that \tool\ achieves an accuracy of 70\% in inferring 77 formal temporal constraints from these sentences.
This paper makes the following main contributions:


\begin{itemize}
	\item An ML and NLP-based approach that effectively infers formal temporal constraints of method invocations. 
	%To the best of our knowledge, our approach is the first one to apply NLP for inferring temporal constraints from API documents.
	\item A prototype implementation of our approach based on extending the Stanford Parser~\cite{Klein03}, which is a natural language parser to derive the grammatical structure of sentences.
	An open source implementation of the prototype is publicly available on our project website\footnote{\url{https://sites.google.com/site/temporalspec}}, along with the experimental subjects and the results. 
	\item An evaluation of our approach on \amazonAPI, \paypalAPI, and commonly used package \CodeIn{java.io} from the JDK API. 
\end{itemize}


The rest of the paper is organized as follows.
Section~\ref{sec:example} presents a real-world example that motivates our approach.
Section~\ref{sec:related} discusses related work.
Section~\ref{sec:background} presents the  background on NLP techniques used in our approach.
Section~\ref{sec:approach} presents our approach.
Section~\ref{sec:evaluation} presents the evaluation of our approach.
Section~\ref{sec:discussion} presents brief discussion on the limitations and future work.
Finally, Section~\ref{sec:conclusion} concludes.

