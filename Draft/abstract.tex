\begin{abstract}


Temporal constraints of an Application Programming Interface (API)
are the allowed sequences of method invocations in the API.
These constraints govern the secure and robust operation of client software using the API.
However, in practice, most API do not have formal temporal constraints.
In contrast, these constraints are typically described
informally in natural language API documents and therefore are not amenable
to existing program analysis tools that require formal constraints.
\textit{The goal of this work is to assist developers construct API clients that comply with 
temporal constraints of the API through the inference and formalization of these constraints found in natural language text in API documents.}
Since, API documents are often verbose manually writing formal temporal constraints can be prohibitively time consuming and error prone.
To address this issue,
we propose \tool, a Machine Learning (ML) and Natural Language Processing (NLP) based
approach to identify and infer formal temporal constraints.
To evaluate our approach, we apply \tool\ to infer and formalize temporal
constraints from \amazonAPI, \paypalAPI\ and commonly used package \CodeIn{java.io} in the
JDK API.
Our results indicate that \tool\ is effective in identifying constraint sentences (from over
4000 human-annotated API sentences) with the average precision, recall, and F-score
of 65.0\%, 72.2\%, and 68.4\%, respectively.
Furthermore, \tool\ also achieves an accuracy of 70\% in inferring 63 formal temporal constraints from these sentences.

%Our approach also includes a novel
%technique of identifying method references (improvement over keyword-based
%techniques) in the natural language text by building domain
%dictionaries systematically from API documents and generic English
%dictionaries.
%
%
%Existing software engineering tools can leverage
%the inferred formal temporal constraints to effectively assist in development
%and verification activities.
	
%Temporal constraints of an Application Programming
%Interface (API) are the allowed sequences of invocations
%of methods from the API.
%These constraints govern the secure and robust
%operation of client software using the API.
%Typically, these constraints are described in natural language text
%of API documents and thus cannot be used by existing verification tools,
%which typically accept only formal constraints.
%Because manually writing temporal constraints based on
%API documents can be prohibitively time consuming and error prone,
%we propose \tool: a Natural Language Processing (NLP) based approach
%to automatically infer temporal constraints.
%In particular, our approach includes a novel technique to reduce the number of lexical tokens
%as a way to make API documents amenable to existing NLP techniques
%(which are typically designed to work on well-written news articles).
%Our approach also includes a novel technique of identifying method references in
%the natural language text by building domain dictionaries systematically
%from API documents and generic English dictionaries.
%To evaluate our approach, we apply \tool\ to infer temporal constraints from 
%commonly used package \CodeIn{java.io} in the JDK API and from the \amazonAPI.
%Our results show that \tool\ effectively identifies constraint sentences (from over 3900 API sentences) with the average precision, recall, and F-score of 65.0\%, 72.2\%, and 68.4\%, respectively.
%Furthermore, \tool\ also achieves an accuracy of 70\% in inferring 63 temporal constraints from these sentences.

% There are existing techniques that infer from API documents resource specifications 
% (such as constraints based on phases: creating, manipulating, and releasing resources).
% However, such techniques are limited to finding temporal specification of resources.
% Furthermore, such techniques do not make a distinction between explicit ordering of methods
% within each phases. 


% where do you loose precision and recall.
\end{abstract}