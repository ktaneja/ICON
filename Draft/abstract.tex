\begin{abstract}
Formal specifications of using a libraries Application Programming Interface (API)
can be used in conjunction with formal analysis tools such as
model checkers and runtime verifiers to assess the quality of a software.
Despite being highly desirable, most API's do not have formal specifications.
In contrast, API documents contain valuable information about the usage in natural language.
However, formal analysis tools are not designed to work on specifications in natural languages.
Manually writing formal specifications based on natural language text in API documents
can be prohibitively time consuming and error prone.
To address this issue, we propose a natural language processing based automated approach \tool ,
to infer formal specifications from natural language text of API documents.
In particular, we focus on temporal constraints, that are defined as constraints on the
\textit{allowed sequence of invocations of methods within an API}.
To evaluate our proposed approach, we applied \tool\ to infer temporal constraints from 
commonly used package \CodeIn{java.io} from JDK API and Amazon S3 REST API.
Our evaluation results show that \tool\ achieves an average of PP\% precision and RR\% recall
in inferring TT temporal constraints from more than ZZZZ sentences of API documents. 


% need to explain why moving from formal specifications to temporal constraints

% Goal statement
% The goal of this research is to facilitate the correct usage API methods by identifying temporal constraints etc.. 
\end{abstract}