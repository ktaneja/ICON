\begin{abstract}
Temporal specifications  or constraints of an Application Programming
Interface (API) describe allowed sequence of invocations
of methods within an API. These constraints govern the secure and robust
operation of client software using these APIs.
Despite being desirable for 
software testing and verification (through formal analysis tools),
most API's do not have formal temporal constraints.
In contrast, API documents contain valuable information (including temporal constraints) about the usage in natural language.
However, manually writing formal constraints based on natural language text in API documents
can be prohibitively time consuming and error prone.


On the other hand, automatically inferring temporal constraints from API documents is challenging.
In particular, it is challenging to automatically 
identify the methods referenced in the natural language text to infer temporal constraints.

% There are existing techniques that infer from API documents resource specifications 
% (such as constraints based on phases: creating, manipulating, and releasing resources).
% However, such techniques are limited to finding temporal specification of resources.
% Furthermore, such techniques do not make a distinction between explicit ordering of methods
% within each phases. 

To address the preceding challenge, we propose a natural-language-processing-based approach \tool,
to infer formal temporal constraints from natural language text of API documents. These constraints can be used (as input)
by various software testing and verification tools to detect defects in client code the APIs.
Our approach 
addresses the preceding challenge by building (and leveraging) domain-dictionaries systematically from API
documents and generic English dictionaries.
To evaluate our proposed approach, we applied \tool\ to infer temporal constraints from 
commonly used package \CodeIn{java.io} from JDK API and \amazonAPI.
Our evaluation results are promising.
In particular, \tool\ achieves an average of 65\% precision and 72\% recall
in inferring 63 temporal constraints from more than 3900 sentences of API documents. 

% Goal Statement
% Negating everything as we go along
% Instead of despite 
% need to explain why moving from formal specifications to temporal constraints
% bridging
% Goal statement
% The goal of this research is to facilitate the correct usage API methods by identifying temporal constraints.
% to support tool based verfication
% Avoid starting the abstarct with a definition
% bridge the gap
% There is an order that API documents tell 
% but they are 
% Adhering to temporal constraints is mandatory for correct usage of API
% However, it not starightforward to 
% Motivation
% Goal
% Evaluation
% Structured Abstract
% what is the significance of this prescison i/p o/p
% expresses 
% makes end to end summary.



% where do you loose precision and recall.
\end{abstract}