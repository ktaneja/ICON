\begin{abstract}


Temporal constraints of an Application Programming Interface (API)
are the allowed sequences of method invocations in the API.
These constraints govern the secure and robust operation of client software using the API.
However, in practice, most APIs do not come with formal temporal constraints.
In contrast, these constraints are typically described
informally in natural language API documents, and therefore are not amenable
to existing tools for checking formal temporal constraints.
\textit{The goal of this work is to assist developers to construct API clients that comply with 
temporal constraints of the API through the inference and formalization of these constraints found in natural language API documents.}
Since API documents are often verbose, manually identifying and writing formal temporal constraints can be prohibitively time-consuming and error-prone.
To address this issue,
we propose \tool: an approach based on Machine Learning (ML) and Natural Language Processing (NLP) for identifying and inferring formal temporal constraints.
To evaluate our approach, we use \tool\ to infer and formalize temporal
constraints from the \amazonAPI, the \paypalAPI, and the \CodeIn{java.io} package in the
JDK API.
Our results indicate that \tool\ can effectively identify temporal
constraint sentences (from over
4000 human-annotated API sentences) with the average precision, recall, and F-score
of 79.0\%, 60.0\%, and 65.0\%, respectively.
Furthermore, our evaluation also demonstrates that \tool\ achieves an accuracy of 70\% in inferring and formalizing 77 temporal constraints from these temporal constraint sentences.

\end{abstract}