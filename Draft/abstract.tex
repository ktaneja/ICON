\begin{abstract}

%\textbf{Objective}: We propose \tool, an automated approach to infer
%temporal constraints from API documents.

\textbf{Objective}: The goal of this work is to assist developers construct
web Application Programming Interface (API) clients that comply with 
temporal constraints of the API thorough the inference and formalization of these constraints found in natural language text in API documents.

\textbf{Background}: Temporal constraints of an API
are the allowed sequences of invocations of
methods from the API. These constraints govern the secure and robust
operation of client software using the API.

\textbf{Method}: \tool, is a Natural Language Processing (NLP) based
approach thus complementing existing program-analysis-based approaches.
In particular, \tool\ includes techniques to reduce the
number of lexical tokens as a way to make API documents amenable
to existing NLP techniques (which are typically designed to work
on well-written news articles). Our approach also includes a novel
technique of identifying method references (improvement over keyword-based
techniques) in the natural language text by building domain
dictionaries systematically from API documents and generic English
dictionaries.

\textbf{Results}: To evaluate our approach, we apply \tool\ to infer temporal
constraints from \amazonAPI, \paypalAPI and commonly used package \CodeIn{java.io} in the
JDK API.
Our results show that \tool\ effectively identifies constraint sentences (from over
3900 API sentences) with the average precision, recall, and F-score
of 65.0\%, 72.2\%, and 68.4\%, respectively. Furthermore, \tool\
also achieves an accuracy of 70\% in inferring 63 temporal constraints
from these sentences.

\textbf{Conclusion}: NLP based techniques offers a viable means to infer
temporal constraints of an API.

\textbf{Application}: Existing software engineering tools can leverage
the inferred temporal constraints to effectively assist in development
and verification activities.
	
%Temporal constraints of an Application Programming
%Interface (API) are the allowed sequences of invocations
%of methods from the API.
%These constraints govern the secure and robust
%operation of client software using the API.
%Typically, these constraints are described in natural language text
%of API documents and thus cannot be used by existing verification tools,
%which typically accept only formal constraints.
%Because manually writing temporal constraints based on
%API documents can be prohibitively time consuming and error prone,
%we propose \tool: a Natural Language Processing (NLP) based approach
%to automatically infer temporal constraints.
%In particular, our approach includes a novel technique to reduce the number of lexical tokens
%as a way to make API documents amenable to existing NLP techniques
%(which are typically designed to work on well-written news articles).
%Our approach also includes a novel technique of identifying method references in
%the natural language text by building domain dictionaries systematically
%from API documents and generic English dictionaries.
%To evaluate our approach, we apply \tool\ to infer temporal constraints from 
%commonly used package \CodeIn{java.io} in the JDK API and from the \amazonAPI.
%Our results show that \tool\ effectively identifies constraint sentences (from over 3900 API sentences) with the average precision, recall, and F-score of 65.0\%, 72.2\%, and 68.4\%, respectively.
%Furthermore, \tool\ also achieves an accuracy of 70\% in inferring 63 temporal constraints from these sentences.

% There are existing techniques that infer from API documents resource specifications 
% (such as constraints based on phases: creating, manipulating, and releasing resources).
% However, such techniques are limited to finding temporal specification of resources.
% Furthermore, such techniques do not make a distinction between explicit ordering of methods
% within each phases. 


% Goal Statement
% Negating everything as we go along
% Instead of despite 
% need to explain why moving from formal specifications to temporal constraints
% bridging
% Goal statement
% The goal of this research is to facilitate the correct usage API methods by identifying temporal constraints.
% to support tool based verfication
% Avoid starting the abstarct with a definition
% bridge the gap
% There is an order that API documents tell 
% but they are 
% Adhering to temporal constraints is mandatory for correct usage of API
% However, it not starightforward to 
% Motivation
% Goal
% Evaluation
% Structured Abstract
% what is the significance of this prescison i/p o/p
% expresses 
% makes end to end summary.



% where do you loose precision and recall.
\end{abstract}